\chapter{lab01}
I dati venivano scambiati attraverso XML, tutte le informazioni dentro a res sono in questo formato. Android studio generalmente li genera, dal momento che dà a disposizione la preview grafica.

\section{Manifest}
Il manifest gestisce i permessi di accesso (localizzazione, ecc.), è obbligatorio e deve assolutamente chiamarsi AndroidManifest.xml. 

\section{Progetto}

\subsection{Gradle}

Android studio sfrutta Gradle per il building, è basato su apache ant e apache maven e ci permette di gestire le dipendenze.

\subsection{Moduli}
Gestione diversa dell'informazione in funzione del device utilizzato (smartphone o android tv o altre robe). Ogni modulo deve contenere: 
\begin{itemize}
	\item manifest
	\item java
	\item res
\end{itemize}

\subsection{Intent-filter}

Va definito in base a ciò che voglio fare, devo specificare l'action della activity presa in considerazione. Una volta che l'app invoca l'intent android va a cercare tutte le app che lo hanno messo disponibile. Ogni activity deve avere un nome specifico ed un package specifico. Quando andiamo a definire le informazioni sono a livello di activity precise, altre a livello di application. 
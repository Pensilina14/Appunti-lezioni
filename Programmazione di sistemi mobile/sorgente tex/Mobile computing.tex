\chapter{Mobile computing}

Il mobile computing è la capacità di calcolo presente ovunque in ogni momento.
\section{Fattori chiave}
\begin{itemize}
	\item Miniaturizzazione $\rightarrow$ Portabilità : Sviluppare componenti sempre più piccoli e potenti;
	\item Connettività : sviluppo di device collegati fra loro (attraverso bluetooth, wifi, ecc.)
\end{itemize}

\section{5G}
La next big thing, ovvero il più grande gomblotto mai creato.

\section{Bluetooth low energy (Beacon)}

Tecnologia che usa il bluetooth con uso ottimale di risorse.

\section{RFID and NFC}

Tecnologie simili, uno permette di scrivere, l'altro agisce solo in lettura.

\section{Convergenza}

I device sono praticamente in grado di fare tutto, come già si sa questo può portare sia vantaggi che svantaggi.

\section{App}

Siamo sommersi da App, è un problema, anche perchè noi siamo abituati ad utilizzarne davvero poche con continuità.

\newpage

\section{Attori principali mobile computing}

\begin{itemize}
	\item Mobile communication;
	\item Hardware mobile;
	\item Software mobile.
\end{itemize}

\section{Sfide mobile}

\begin{itemize}
	\item Resource-constrained: i dispositivi, come sappiamo hanno una durata limitata della batteria;
	\item Connettività mobile;
	\item Sicurezza e privacy.
\end{itemize}

\section{Sviluppo nativo e ibrido}

Ibrido : sviluppo una volta e incastro nei due sistemi operativi.\\
Nativo : sviluppo unicamente per un sistema operativo.

\begin{lstlisting}
	//Hello.java
	import javax.swing.JApplet;
	import java.awt.Graphics;
	
	public class Hello extends JApplet {
		public void paintComponent(Graphics g) {
			g.drawString("Hello, world!", 65, 95);
		}    
	}
\end{lstlisting}
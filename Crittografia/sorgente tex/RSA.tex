\chapter{RSA}

\section{Introduzione}

Rivoluzione nel campo della crittografia, stiamo parlando di tecniche fortemente utilizzate al giorno d'oggi, ovvero di protocolli a chiave pubblica. Permette di condividere un segreto fra mittente e destinatario, nonostante non esista una chiave posso comunicare in completa segretezza. Questa rivoluzione è avvenuta negli anni 70.

\section{Descrizione}

Nei cifrari simmetrici visti sinora la chiave di cifratura è
uguale a quella di decifrazione (o comunque ciascuna può
essere facilmente calcolata dall’altra), ed è nota solo ai
due partner che la scelgono di comune accordo e la
mantengono segreta. Nei cifrari a chiave pubblica, o asimmetrici, le chiavi di cifratura e di decifrazione sono completamente diverse tra
loro. Esse sono scelte dal destinatario che rende pubblica
la chiave di cifratura k[pub], che è quindi nota a tutti, e
mantiene segreta la chiave di decifrazione k[prv] che è
quindi nota soltanto a chi la genera. Chiunque voglia spedirmi qualcosa deve conoscere la mia chiave pubblica. Esiste una coppia k[pub], k[prv] per ogni utente del
sistema, scelta da questi nella sua veste di possibile
destinatario Dest. La cifratura di un messaggio m da
inviare a Dest è eseguita da qualunque mittente come
c = C(m; k[pub]), ove sia la chiave k[pub] che la funzione
di cifratura C sono note a tutti. La decifrazione è eseguita
da Dest come m = D(c; k[prv]), ove D è la funzione di
decifrazione anch’essa nota a tutti, ma k[prv] non è
disponibile agli altri che non possono quindi ricostruire m. Il concetto base era già noto nel 76, non era invece chiara quale C e quale D avessero queste proprietà.

\section{Asimmetria}

L’appellativo di asimmetrici assegnato a questi cifrari
sottolinea i ruoli completamente diversi svolti da Mitt e
Dest, in contrapposizione ai ruoli intercambiabili che essi
hanno nei cifrari simmetrici ove condividono la stessa
informazione (cioè la chiave) segreta. 

\newpage

\section{Proprietà}

\begin{itemize}
	\item Per ogni possibile messaggio $m$ si ha:\\
	$D(C(m, k[pub]), k[prv]) = m$ \\
	Dunque stiamo dicendo che $D(C(m)) = m $, ossia Dest deve avere la possibilità di interpretare qualunque messaggio che gli altri utenti decidano di spedirgli. $C(D(m)) = m$ invece non vale sempre, chi soddisfa questa proprietà commutativa è molto apprezzato, perchè permette di implementare la firma digitale.;
	\item La sicurezza e l’efficienza del sistema dipendono dalle
	funzioni C e D, e dalla relazione che esiste tra le chiavi
	k[prv] e k[pub] di ogni coppia;
	\item La coppia k[prv] e k[pub] è facile da generare, e deve
	risultare praticamente impossibile che due utenti scelgano
	la stessa chiave deve essere inoltre praticamente impossibile (computazionalmente difficile) risalire alla chiave privata dalla pubblica.
	\item Dati m e k[pub], è facile per il mittente calcolare il
	crittogramma c = C(m, k[pub]). (deve essere facile la codifica).
	\item Dati c e k[prv], è facile per il destinatario calcolare il
	messaggio originale m = D(c, k[prv]) (deve essere facile la decodifica)
	\item Pur conoscendo il crittogramma c, la chiave pubblica
	k[pub], e le funzioni C e D, è difficile per un
	crittoanalista risalire al messaggio m, ancora peggio alla chiave privata!. Risalire al messaggio smaschera una comunicazione (violazione puntuale), risalire alla chiave privata comporta invece una violazione totale del protocollo.
\end{itemize}

\section{Sicurezza}

I protocolli a chiave privata vengono definiti sicuri in maniera empirica, ergo se non è stato violato in n anni di utilizzo è sicuro. Nei protocolli a chiave pubblica invece la sicurezza è determinata dal fatto che per violarli dovremmo riuscire a risolvere in tempo polinomiale dei problemi che fino ad esso non siamo riusciti a risolvere (non è stato possibile risolvere questo problema in anni,secoli e millenni, per questo è un certificato di garanzia di gran lunga migliore rispetto a quello dei protocolli a chiave privata).

\section{Funzioni one-way}

C deve essere una funzione one-way, cioè facile da calcolare e difficile da invertire, ma deve contenere un meccanismo segreto detto trap-door che ne consenta la facile invertibilità solo a chi conosca tale meccanismo (questo è il succo, la sintesi dei protocolli a chiave pubblica, lo svela-segreto è la chiave privata). La conoscenza di k[pub] non fornisce alcuna indicazione sul meccanismo segreto, che è svelato da k[prv] quando questa chiave è inserita nella funzione D (D = decodifica).

\section{Richiami di algebra modulare}

$\mathbb{Z}_n$ è l'insieme di numeri modulo $m$. $\mathbb{Z}^{*}_n$ è un sottoinsieme di $\mathbb{Z}_n$ e rappresenta l'insieme di elementi di $\mathbb{Z}_n$ relativamente primi con $n$, 0 ed 1 esclusi. La notazione $a$ mod $n$ significa fare a fette $a$ in chunk lunghi $n$. Se prendo due numeri e li divido per lo stesso $n$ e ottengo lo stesso resto sono nella stessa classe di equivalenza.

\begin{itemize}
	\item $\mathbb{Z}_n = \{0, 1, ..., n\}$;
	\item $\mathbb{Z}^{*}_p = \{1, ..., p - 1\}$;
	\item $a$ mod $n$ è il resto della divisione intera fra $a$ e $n$;
	\item $a$ $\cong$ $b$ mod $n$ se e solo se $a$ mod $n$ = $b$ mod $n$;
	\item $a$ $\cong$ $b$ mod $n$ se e solo se $a$ = $b + kn$; 
	\item $a$ primo con $b$ se e solo se il loro massimo comun divisore è uguale a 1.
\end{itemize}

\section{Funzione di Eulero}

Concetto importantissimo, non fa altro che dirci quanto è grande $\mathbb{Z}^{*}_n$, ovvero la sua cardinalità.Ovvero:\\\\
$\phi(n) = |\mathbb{Z}^{*}_n|$
\subsection{Proprietà}
\begin{itemize}
	\item $\phi(p) = p - 1$ 
	\item $\phi(pq) = (p - 1) (q - 1)$
	\item $a$ primo con $b$ allora $\phi(ab) = \phi(a) \phi(b)$
	\item $\phi(n)$ difficile da calcolare (non riusciamo a calcolarla efficientemente, se sapessimo fattorizzare sarebbe facile anche calcolare $\phi$. Anche se fosse difficile fattorizzare noi potremmo tentare di trovare $\phi$ con metodi alternativi).
\end{itemize}

\section{Teorema di Eulero}

Sia $n > 1$ e $a$ primo con $n$, allora:\\\\
$a^{\phi(n)} \cong 1$ mod $n$ 